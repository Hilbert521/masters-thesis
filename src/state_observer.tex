\section{State observer}

Implementing an advanced control mechanism such as \textit{linear quadratic controller}, \textit{full state feedback} or in our case, \textit{model predictive controller}, requires a knowledge of all system states. This is a different situation then in a case of previously used PID controller, when only the controlled state needs to be known. There are some situations and corresponding systems, where the knowledge of all states is relatively easy to reach, respectively it might be already measured for another reason. For instance space, aerials or another vehicles, where the typical states are position and its derivatives. But in the case where not all states can be measured or we do not want to measure them, the state observer needs to be implemented. The state observer is a dynamical system which is simulated concurrently with the controlled system. It has the same number of inputs and it is controlled by the same actions as the real system. The order of the observer is the same as the order of the system. Its states are supposed to be observable and should correspond to the real system's states.

\subsection{Open-loop observer}

Let us have an LTI system that needs to be observed (considering $\textbf{C} = \mathbb{I}$, $\textbf{D} = \mathbf{0}$)

\begin{equation}
\textbf{x} = \textbf{A}\textbf{x} + \textbf{B}\textbf{u} + \textbf{w}
\end{equation}

The Open-loop observer can be constructed by setting up a system

\begin{equation}
\textbf{\^x} = \textbf{A}\textbf{\^x} + \textbf{B}\textbf{u}
\end{equation}

Following situations can be expected during the execution

\begin{itemize}
\item Estimated states track well corresponding system states. In this situation the system was identified perfectly and there is no need for feedback control.
\item Estimated states track well system sates, but there is some drift during the time frame of the experiment.
\item States estimated by the observer are completely out of scope of the system states which may suggest that the observer's model is wrong.
\end{itemize}

Open-loop estimations in chapter \ref{cap:system_identification} indicate that the model is in our case satisfactory but the Open-loop estimator would not probably lead to precise control results.

\subsection{Closed-loop observer}

The open-loop observer can be basically corrected by closing a feedback loop around the system.

\begin{equation}
\textbf{\^x} = \textbf{A}\textbf{\^x} + \textbf{B}\textbf{u} - \textbf{L}\left(\textbf{x} - \textbf{\^x}\right)
\end{equation}

There are methods of finding $\textbf{L}$ such that poles of the observer are properly placed\footnote{Assuming the system can be converted to the \textit{controlled canonical form}}. One can utilize that placing poles of a state feedback is a dual problem for placing poles of the observer.

\subsection{Kalman filter}

For the purpose of this Thesis, the Kalman filter was implemented to estimate all states of the helicopter and to filter the measured data from sensors. The Kalman filter is a closed-loop iterative estimator. A hypothesis about estimated states takes form of a normal distribution with mean vector $\hat{\textbf{x}}$ and covariance matrix $\hat{\boldsymbol{\Sigma}}$. It presumes a model in a following form

\begin{equation}
\textbf{x}[t+1] = \textbf{A}\textbf{x}[t] + \textbf{B}\textbf{u}[t] + \textbf{w}[t]
\end{equation}

where \textbf{w} is a process noise vector. The noise is supposed to be drawn from a zero-mean normal distribution with a covariance matrix \textbf{R}. Further due there is a sensor model

\begin{equation}
\textbf{z}[t] = \textbf{H}\textbf{x}[t] + \textbf{v}[t]
\end{equation}

where \textbf{z}[t] is a measurement, \textbf{H} is a matrix that maps the state vector to the measurement and \textbf{v}[t] is the measurement noise vector. The noise is again drawn from a normal distribution with a covariance matrix \textbf{Q}. The filter holds a state vector $\textbf{\^x}$ and its covariance matrix $\boldsymbol{\Omega}$ between iterations. There are two phases of the Kalman filter's algorithm. The first one is the \textit{prediction phase}, where new state vector is estimated using a model. Its covariance $\boldsymbol{\Omega}$ is modified using a process noise covariance $\textbf{R}$. The second phase is the \textit{correction phase} when the state vector is updated using a measured data and $\boldsymbol{\Sigma}$ is again modified using the noise measurement covariance \textbf{P}.      

\subsubsection*{Prediction phase}

\begin{equation}
\begin{split}
\hat{\textbf{x}}[t] &\leftarrow \textbf{A}\hat{\textbf{x}}[t-1] + \textbf{B}\textbf{u}[t] \\
\hat{\boldsymbol{\Sigma}}[t] &\leftarrow \textbf{A}\hat{\boldsymbol{\Sigma}}[t-1]\textbf{A}^{T} + \textbf{R}
\end{split}
\end{equation}

\subsubsection*{Correction phase}

\begin{equation}
\begin{split}
\textbf{K}[t] &\leftarrow \hat{\boldsymbol{\Sigma}}[t]\textbf{C}^{T}\left(\textbf{C}\hat{\boldsymbol{\Sigma}}[t]\textbf{C}^{T} + \textbf{Q}\right)^{-1} \\
\hat{\textbf{x}}[t] &\leftarrow \hat{\textbf{x}}[t] + \textbf{K}[t]\left(\textbf{z}[t] - \textbf{C}\hat{\textbf{x}}[t]\right) \\
\hat{\boldsymbol{\Sigma}}[t] &\leftarrow \left(\mathbb{I} - \textbf{K}[t]\textbf{C}\right)\hat{\boldsymbol{\Sigma}}
\end{split}
\end{equation}

The prediction phase can be easily compared to the open-loop observer. 