\section{UAV dynamics}

Modeling a system dynamics has always been a great part of a control design process. One with a good understanding of a system's behavior can design a well suited controller using many approaches taught in the field of control theory. Such controller can reflect known characteristics of the system and can appropriately react to evolution of system states. There are two fundamental approaches to the system modeling. One is based on the knowledge about the physics involved in the system. Such knowledge can be used to derive a mathematical model using well known principles, e.g. Hamiltonian mechanics. Control design for such system is usually called \strong{WhiteBox}, or \strong{GreyBox}, depending on how much of the physical process we are able to describe. On the other hand, when the system is unknown, one can create a mathematical model which sufficiently represents observed behavior of the system. Such method is usually called \strong{BlackBox}.

When modeling a vehicle such as classical helicopter, we could create a complex model including phenomena as aerodynamics, rotor-blade flapping and other. In many cases~\citep{alexis2014robust}\citep{mahony2012multirotor}, dynamics of a multirotor MAV can be simplified to a single rigid-body methods. Further due to existence of well designed and tested platforms such as Pixhawk~\citep{pixhawk} or Ardupilot~\citep{ardupilot} we can model the vehicle with the inner feedback loop closed. By doing that, considering a fixed-pitch quadcopter, we move from a system actuated by four thrust values to a system, where the inputs are the desired pitch ($\theta$), roll ($\psi$), yaw rate ($\dot{\phi}$) and collective thrust ($T$). 
